\PassOptionsToPackage{unicode=true}{hyperref} % options for packages loaded elsewhere
\PassOptionsToPackage{hyphens}{url}
%
\documentclass[
]{article}
\usepackage{lmodern}
\usepackage{amssymb,amsmath}
\usepackage{ifxetex,ifluatex}
\ifnum 0\ifxetex 1\fi\ifluatex 1\fi=0 % if pdftex
  \usepackage[T1]{fontenc}
  \usepackage[utf8]{inputenc}
  \usepackage{textcomp} % provides euro and other symbols
\else % if luatex or xelatex
  \usepackage{unicode-math}
  \defaultfontfeatures{Scale=MatchLowercase}
  \defaultfontfeatures[\rmfamily]{Ligatures=TeX,Scale=1}
\fi
% use upquote if available, for straight quotes in verbatim environments
\IfFileExists{upquote.sty}{\usepackage{upquote}}{}
\IfFileExists{microtype.sty}{% use microtype if available
  \usepackage[]{microtype}
  \UseMicrotypeSet[protrusion]{basicmath} % disable protrusion for tt fonts
}{}
\makeatletter
\@ifundefined{KOMAClassName}{% if non-KOMA class
  \IfFileExists{parskip.sty}{%
    \usepackage{parskip}
  }{% else
    \setlength{\parindent}{0pt}
    \setlength{\parskip}{6pt plus 2pt minus 1pt}}
}{% if KOMA class
  \KOMAoptions{parskip=half}}
\makeatother
\usepackage{xcolor}
\IfFileExists{xurl.sty}{\usepackage{xurl}}{} % add URL line breaks if available
\IfFileExists{bookmark.sty}{\usepackage{bookmark}}{\usepackage{hyperref}}
\hypersetup{
  pdftitle={Assignment 1},
  pdfauthor={Walter Verwer \& Bas Machielsen},
  pdfborder={0 0 0},
  breaklinks=true}
\urlstyle{same}  % don't use monospace font for urls
\usepackage[margin=1in]{geometry}
\usepackage{color}
\usepackage{fancyvrb}
\newcommand{\VerbBar}{|}
\newcommand{\VERB}{\Verb[commandchars=\\\{\}]}
\DefineVerbatimEnvironment{Highlighting}{Verbatim}{commandchars=\\\{\}}
% Add ',fontsize=\small' for more characters per line
\usepackage{framed}
\definecolor{shadecolor}{RGB}{248,248,248}
\newenvironment{Shaded}{\begin{snugshade}}{\end{snugshade}}
\newcommand{\AlertTok}[1]{\textcolor[rgb]{0.94,0.16,0.16}{#1}}
\newcommand{\AnnotationTok}[1]{\textcolor[rgb]{0.56,0.35,0.01}{\textbf{\textit{#1}}}}
\newcommand{\AttributeTok}[1]{\textcolor[rgb]{0.77,0.63,0.00}{#1}}
\newcommand{\BaseNTok}[1]{\textcolor[rgb]{0.00,0.00,0.81}{#1}}
\newcommand{\BuiltInTok}[1]{#1}
\newcommand{\CharTok}[1]{\textcolor[rgb]{0.31,0.60,0.02}{#1}}
\newcommand{\CommentTok}[1]{\textcolor[rgb]{0.56,0.35,0.01}{\textit{#1}}}
\newcommand{\CommentVarTok}[1]{\textcolor[rgb]{0.56,0.35,0.01}{\textbf{\textit{#1}}}}
\newcommand{\ConstantTok}[1]{\textcolor[rgb]{0.00,0.00,0.00}{#1}}
\newcommand{\ControlFlowTok}[1]{\textcolor[rgb]{0.13,0.29,0.53}{\textbf{#1}}}
\newcommand{\DataTypeTok}[1]{\textcolor[rgb]{0.13,0.29,0.53}{#1}}
\newcommand{\DecValTok}[1]{\textcolor[rgb]{0.00,0.00,0.81}{#1}}
\newcommand{\DocumentationTok}[1]{\textcolor[rgb]{0.56,0.35,0.01}{\textbf{\textit{#1}}}}
\newcommand{\ErrorTok}[1]{\textcolor[rgb]{0.64,0.00,0.00}{\textbf{#1}}}
\newcommand{\ExtensionTok}[1]{#1}
\newcommand{\FloatTok}[1]{\textcolor[rgb]{0.00,0.00,0.81}{#1}}
\newcommand{\FunctionTok}[1]{\textcolor[rgb]{0.00,0.00,0.00}{#1}}
\newcommand{\ImportTok}[1]{#1}
\newcommand{\InformationTok}[1]{\textcolor[rgb]{0.56,0.35,0.01}{\textbf{\textit{#1}}}}
\newcommand{\KeywordTok}[1]{\textcolor[rgb]{0.13,0.29,0.53}{\textbf{#1}}}
\newcommand{\NormalTok}[1]{#1}
\newcommand{\OperatorTok}[1]{\textcolor[rgb]{0.81,0.36,0.00}{\textbf{#1}}}
\newcommand{\OtherTok}[1]{\textcolor[rgb]{0.56,0.35,0.01}{#1}}
\newcommand{\PreprocessorTok}[1]{\textcolor[rgb]{0.56,0.35,0.01}{\textit{#1}}}
\newcommand{\RegionMarkerTok}[1]{#1}
\newcommand{\SpecialCharTok}[1]{\textcolor[rgb]{0.00,0.00,0.00}{#1}}
\newcommand{\SpecialStringTok}[1]{\textcolor[rgb]{0.31,0.60,0.02}{#1}}
\newcommand{\StringTok}[1]{\textcolor[rgb]{0.31,0.60,0.02}{#1}}
\newcommand{\VariableTok}[1]{\textcolor[rgb]{0.00,0.00,0.00}{#1}}
\newcommand{\VerbatimStringTok}[1]{\textcolor[rgb]{0.31,0.60,0.02}{#1}}
\newcommand{\WarningTok}[1]{\textcolor[rgb]{0.56,0.35,0.01}{\textbf{\textit{#1}}}}
\usepackage{graphicx,grffile}
\makeatletter
\def\maxwidth{\ifdim\Gin@nat@width>\linewidth\linewidth\else\Gin@nat@width\fi}
\def\maxheight{\ifdim\Gin@nat@height>\textheight\textheight\else\Gin@nat@height\fi}
\makeatother
% Scale images if necessary, so that they will not overflow the page
% margins by default, and it is still possible to overwrite the defaults
% using explicit options in \includegraphics[width, height, ...]{}
\setkeys{Gin}{width=\maxwidth,height=\maxheight,keepaspectratio}
\setlength{\emergencystretch}{3em}  % prevent overfull lines
\providecommand{\tightlist}{%
  \setlength{\itemsep}{0pt}\setlength{\parskip}{0pt}}
\setcounter{secnumdepth}{-2}
% Redefines (sub)paragraphs to behave more like sections
\ifx\paragraph\undefined\else
  \let\oldparagraph\paragraph
  \renewcommand{\paragraph}[1]{\oldparagraph{#1}\mbox{}}
\fi
\ifx\subparagraph\undefined\else
  \let\oldsubparagraph\subparagraph
  \renewcommand{\subparagraph}[1]{\oldsubparagraph{#1}\mbox{}}
\fi

% set default figure placement to htbp
\makeatletter
\def\fps@figure{htbp}
\makeatother


\title{Assignment 1}
\author{Walter Verwer \& Bas Machielsen}
\date{\today}

\begin{document}
\maketitle

\hypertarget{question-1-the-sample-selection-model.}{%
\subsection{Question 1: The sample selection
model.}\label{question-1-the-sample-selection-model.}}

A researcher aims to gain insight in the potential earnings of the
non-employed. (In the data, the non-employed can be identified by a
missing value for the earnings variable). She realizes that the sample
of observed wages may be subject to sample selection.

\textbf{(a) Run an OLS regression for log-earnings on schooling, age,
and age squared. Present the results and comment on the estimates.}

\begin{Shaded}
\begin{Highlighting}[]
\NormalTok{model_}\DecValTok{1}\NormalTok{ <-}\StringTok{ }\KeywordTok{lm}\NormalTok{(}\DataTypeTok{data =}\NormalTok{ data, }\DataTypeTok{formula =}\NormalTok{ logWage }\OperatorTok{~}\StringTok{ }\NormalTok{schooling }\OperatorTok{+}\StringTok{ }\NormalTok{age }\OperatorTok{+}\StringTok{ }\NormalTok{age2)}

\NormalTok{results_}\DecValTok{1}\NormalTok{ <-}\StringTok{ }\KeywordTok{summary.lm}\NormalTok{(model_}\DecValTok{1}\NormalTok{)}
\NormalTok{coeffs_}\DecValTok{1}\NormalTok{ <-}\StringTok{ }\NormalTok{results_}\DecValTok{1}\OperatorTok{$}\NormalTok{coefficients[,}\DecValTok{1}\NormalTok{]}
\NormalTok{pvals_}\DecValTok{1}\NormalTok{ <-}\StringTok{ }\NormalTok{results_}\DecValTok{1}\OperatorTok{$}\NormalTok{coefficients[,}\DecValTok{4}\NormalTok{]}

\KeywordTok{stargazer}\NormalTok{(model_}\DecValTok{1}\NormalTok{, }\DataTypeTok{style =} \StringTok{"AER"}\NormalTok{, }
          \DataTypeTok{font.size =} \StringTok{"small"}\NormalTok{,}
          \DataTypeTok{header =}\NormalTok{ F, }\DataTypeTok{label =} \StringTok{'tab:q1_a_ols'}\NormalTok{, }
          \DataTypeTok{notes =} \StringTok{'OLS regression for log-earnings on schooling, age and age squared.'}\NormalTok{)}
\end{Highlighting}
\end{Shaded}

\begin{table}[!htbp] \centering 
  \caption{} 
  \label{tab:q1_a_ols} 
\small 
\begin{tabular}{@{\extracolsep{5pt}}lc} 
\\[-1.8ex]\hline 
\hline \\[-1.8ex] 
\\[-1.8ex] & logWage \\ 
\hline \\[-1.8ex] 
 schooling & 0.216$^{***}$ \\ 
  & (0.032) \\ 
  & \\ 
 age & $-$0.342 \\ 
  & (0.521) \\ 
  & \\ 
 age2 & $-$0.011 \\ 
  & (0.008) \\ 
  & \\ 
 Constant & 26.400$^{***}$ \\ 
  & (8.060) \\ 
  & \\ 
Observations & 416 \\ 
R$^{2}$ & 0.815 \\ 
Adjusted R$^{2}$ & 0.813 \\ 
Residual Std. Error & 1.500 (df = 412) \\ 
F Statistic & 604.000$^{***}$ (df = 3; 412) \\ 
\hline \\[-1.8ex] 
\textit{Notes:} & \multicolumn{1}{l}{$^{***}$Significant at the 1 percent level.} \\ 
 & \multicolumn{1}{l}{$^{**}$Significant at the 5 percent level.} \\ 
 & \multicolumn{1}{l}{$^{*}$Significant at the 10 percent level.} \\ 
 & \multicolumn{1}{l}{OLS regression for log-earnings on schooling, age and age squared.} \\ 
\end{tabular} 
\end{table}

The results show that 1 additional year of schooling has an effect of
0.216 on log(Wage), which means that 1 additional year of schooling has
an estimated 21.6\% effect on wages earned. This result is highly
significant (p-value is \ensuremath{2.706\times 10^{-11}}) Another
result shown in figure \ref{tab:q1_a_ols} is that being 1 year older has
an estimated effect of -0.342 on log(Wage). Thus, being 1 year older is
estimated to have a -34.189\% on wages. This result is not significant
at a common level (p-value is 0.512). The third estimated coefficients
is the one corresponding to the variable age squared. It has an
estimated coefficient of -0.011, which represents the estimated effect
of a one unit increase in age squared on log(Wage). This also means that
a one unit increase in age squared is equal to an estimate effect of
-1.114\% on the linear representation of wage. This effect is not
significant at common levels, because the p-value is 0.184. Finally, the
estimated constant in the model is estimated at 26.409. This means that
if all other coefficients are 0, then the log(Wage) will be equal to
26.409. Thus, if all other variables are 0, then wage is estimated to be
equal to exp(26.409)=\ensuremath{2.947\times 10^{11}}. This is an
extremely high number given the characteristics of the wage variable.
However, this result is highly significant, because the p-value is
0.001.

\textbf{(b) Briefly discuss the sample selection problem that may arise
in using these OLS estimates for the purpose of predicting the potential
earnings of the non-employed. Formulate the sample selection model. In
your answer, include an explanation why OLS may fail in this context.}

An individual is only in this dataset if they earn wages, i.e.~if they
are employed. Being employed itself is not randomly allocated, but
rather, a function of e.g.~age, age\^{}2, and schooling. Hence, the
estimates of schooling on earnings are conditional on having earnings to
begin with, whereas unbiased estimates must also include those
individuals

Formally,
\(\mathbb{E}[\text{Earnings}] = \mathbb{E[\text{Earnings}|\text{Having a job}]} \cdot \mathbb{P}[\text{Having a job}] + \mathbb{E[\text{Earnings}|\text{Not having a job}]} \cdot (1-\mathbb{P[\text{Having a job}]})\).
The given estimation only concerns
\(\mathbb{E[\text{Earnings}|\text{Having a job}]}\).

\textbf{(c) Which variable in your data may be a suitable candidate as
an exclusion restriction for the sample selection model?}

\hypertarget{question-2-earnings-and-schooling}{%
\subsection{Question 2: Earnings and
Schooling}\label{question-2-earnings-and-schooling}}

The same researcher is interested in estimating the causal effect of
schooling on earnings for employed individuals only. As a consequence,
she performs the subsequent analysis on the (sub)sample of employed
individuals.

\textbf{(a) Discuss the estimation of the causal effect of schooling on
earnings by OLS. In particular, address whether or not it is plausible
that regularity conditions for applying OLS are satisfied.}

It is not plausible that the regularity conditions are satisfied. In
particular, an observable such as an individual's \textbf{ability} might
be correlated with the wage, but also with the decision to live close to
a school. Hence, the estimates suffer from endogeneity.

\textbf{(b) The researcher has collected data on two potential
instrumental variables subsidy and distance for years of schooling.}

\begin{itemize}
\tightlist
\item
  \textbf{distance measures the distance between the school location and
  the residence of the individual while at school-going age.}
\item
  \textbf{subsidy is an indicator depending on regional subsidies of
  families for covering school expenses.}
\end{itemize}

The researcher has the option to use only distance as an instrumental
variable, or to use only the instrumental variable subsidy, or to use
both distance and subsidy as instrumental variables. Perform
instrumental variables estimation for these three options. Which option
do you prefer? Include in your answer the necessary analyses and numbers
on which you base your choice.

\begin{Shaded}
\begin{Highlighting}[]
\NormalTok{firstoption <-}\StringTok{ }\KeywordTok{ivreg}\NormalTok{(}\DataTypeTok{data =}\NormalTok{ data, }\DataTypeTok{formula =} 
\NormalTok{                         logWage }\OperatorTok{~}\StringTok{ }\NormalTok{age }\OperatorTok{+}\StringTok{ }\NormalTok{age2 }\OperatorTok{+}\StringTok{ }\NormalTok{schooling }\OperatorTok{|}\StringTok{ }\NormalTok{distance }\OperatorTok{+}\StringTok{ }\NormalTok{age }\OperatorTok{+}\StringTok{ }\NormalTok{age2)}

\NormalTok{secondoption <-}\StringTok{ }\KeywordTok{ivreg}\NormalTok{(}\DataTypeTok{data =}\NormalTok{ data, }\DataTypeTok{formula =}
\NormalTok{                          logWage }\OperatorTok{~}\StringTok{ }\NormalTok{age }\OperatorTok{+}\StringTok{ }\NormalTok{age2 }\OperatorTok{+}\StringTok{ }\NormalTok{schooling }\OperatorTok{|}\StringTok{ }\NormalTok{subsidy }\OperatorTok{+}\StringTok{ }\NormalTok{age }\OperatorTok{+}\StringTok{ }\NormalTok{age2)}

\NormalTok{thirdoption <-}\StringTok{ }\KeywordTok{ivreg}\NormalTok{(}\DataTypeTok{data =}\NormalTok{ data, }\DataTypeTok{formula =}
\NormalTok{                          logWage }\OperatorTok{~}\StringTok{ }\NormalTok{age }\OperatorTok{+}\StringTok{ }\NormalTok{age2 }\OperatorTok{+}\StringTok{ }\NormalTok{schooling }\OperatorTok{|}\StringTok{ }\NormalTok{subsidy }\OperatorTok{+}\StringTok{ }\NormalTok{distance }\OperatorTok{+}\StringTok{ }\NormalTok{age }
                     \OperatorTok{+}\StringTok{ }\NormalTok{age2)}
\end{Highlighting}
\end{Shaded}

\begin{Shaded}
\begin{Highlighting}[]
\KeywordTok{stargazer}\NormalTok{(firstoption, secondoption, thirdoption, }\DataTypeTok{font.size =} \StringTok{"small"}\NormalTok{, }
          \DataTypeTok{style =} \StringTok{"AER"}\NormalTok{, }
          \DataTypeTok{header =}\NormalTok{ F)}
\end{Highlighting}
\end{Shaded}

\begin{table}[!htbp] \centering 
  \caption{} 
  \label{} 
\small 
\begin{tabular}{@{\extracolsep{5pt}}lccc} 
\\[-1.8ex]\hline 
\hline \\[-1.8ex] 
\\[-1.8ex] & \multicolumn{3}{c}{logWage} \\ 
\\[-1.8ex] & (1) & (2) & (3)\\ 
\hline \\[-1.8ex] 
 age & $-$0.192 & $-$0.233 & $-$0.229 \\ 
  & (0.587) & (0.546) & (0.547) \\ 
  & & & \\ 
 age2 & $-$0.014 & $-$0.013 & $-$0.013 \\ 
  & (0.010) & (0.009) & (0.009) \\ 
  & & & \\ 
 schooling & 0.470 & 0.401$^{***}$ & 0.408$^{***}$ \\ 
  & (0.299) & (0.106) & (0.102) \\ 
  & & & \\ 
 Constant & 22.700$^{**}$ & 23.700$^{***}$ & 23.600$^{***}$ \\ 
  & (9.700) & (8.520) & (8.530) \\ 
  & & & \\ 
Observations & 416 & 416 & 416 \\ 
R$^{2}$ & 0.786 & 0.799 & 0.798 \\ 
Adjusted R$^{2}$ & 0.784 & 0.798 & 0.797 \\ 
Residual Std. Error (df = 412) & 1.610 & 1.560 & 1.560 \\ 
\hline \\[-1.8ex] 
\textit{Notes:} & \multicolumn{3}{l}{$^{***}$Significant at the 1 percent level.} \\ 
 & \multicolumn{3}{l}{$^{**}$Significant at the 5 percent level.} \\ 
 & \multicolumn{3}{l}{$^{*}$Significant at the 10 percent level.} \\ 
\end{tabular} 
\end{table}

We consider that the second option, to include only subsidy as an
instrument, is the best option. The reason is that distance is unlikely
to satisfy the exclusion restriction: distance is (to a certain extent)
an endogenous variable: wealthier (or more able) parents may choose to
live closer to school, and invest more in the education of their
children (or genetically transmit ability). Since a potentially
endogenous instrument must not be used as such, we prefer the estimates
in equation 2. However, we see that the results show that distance has
no predictive power in schooling, thus showing that the endogeneity is
very small. Conditional on subsidy being a good instrument, then, the
potential endogeneity does not substantially changes the estimates of
schooling on earnings.

\begin{enumerate}
\def\labelenumi{(\alph{enumi})}
\setcounter{enumi}{2}
\tightlist
\item
  Compare the IV estimates with the OLS outcomes. Under which conditions
  would you prefer OLS over IV? Perform a test and use the outcome of
  the test to support your choice between OLS and IV. Motivate your
  choice.
\end{enumerate}

We first observe that the OLS estimate \(\beta =\) 0.216 is about half
the magnitude of the IV-estimate. This means that the bias generated by
OLS likely \emph{downplays} the actual effect (if the IV estimates
satisfy the exclusion restriction). In case we would not trust the IV
assumptions, we would prefer to trust the (conservative) estimate that
downplays the effect, i.e.~the OLS estimates. We can test whether the
OLS estimates are substantially different from the IV estimates by
conducting a Hausman test:

\begin{Shaded}
\begin{Highlighting}[]
\NormalTok{hoi <-}\StringTok{ }\KeywordTok{summary}\NormalTok{(}
\NormalTok{    secondoption, }
        \DataTypeTok{diagnostics=}\OtherTok{TRUE}\NormalTok{)}

\NormalTok{hoi}\OperatorTok{$}\NormalTok{diagnostics}
\end{Highlighting}
\end{Shaded}

\begin{verbatim}
##                  df1 df2 statistic  p-value
## Weak instruments   1 412     43.32 1.42e-10
## Wu-Hausman         1 411      3.63 5.73e-02
## Sargan             0  NA        NA       NA
\end{verbatim}

The null hypothesis in the Hausman test is exogeneity of the
\emph{schooling} variable. As becomes clear, the null hypothesis is
marginally rejected, implying the \emph{schooling} is endogenous, but
only marginally so. Hence, we would prefer to trust the IV estimates in
this case.

\end{document}
